\documentclass[10pt,a4paper]{article}
\usepackage[utf8]{inputenc}
\usepackage[english]{babel}
\usepackage[english]{isodate}
\usepackage[parfill]{parskip}
\usepackage{url}
\usepackage{todonotes}
\usepackage{amsmath}

\usepackage{tikz}
\usepackage{signalflowdiagram}
\usetikzlibrary{arrows,shapes,positioning,matrix,chains,calc}
\usepackage{pgfplots}

\setlength{\topmargin}{-.5in}
\setlength{\textheight}{9in}
\setlength{\oddsidemargin}{.125in}
\setlength{\textwidth}{6.25in}

\usepackage{listings}

%%%

\begin{document}
\title{Simni Short Manual}
\author{Stefan Bethge\\
Neurorobotics Research Laboratory, Humbold-Universität zu Berlin}

\pagenumbering{Roman}
\maketitle
\tableofcontents
\newpage
\pagenumbering{arabic}

\section{Introduction}

The Simni simulator is a testbed for simple robots that work in a two-dimensional world (restricted). The built in morphology is imitating the Semni morphology of the Neurorobotics Research Laboratory, Berlin, (see \url{http://neurorobotics.de}).
On top of the physics simulation, it contains an implementation of the ABC learning framework to explore the morphology's own stable and instable fixpoints and hence find trajectories between energy efficient postures. 

It provides phase space and motor voltage plots and a graph representation of the interesting postures and connection between them in the explored manifold.

\section{Requirements}

\begin{itemize}
\item A browser, preferrably with a fast JavaScript implementation like V8. Recommended at the time of writing is Google Chrome or Chromium > 30
\item For compiling the CoffeeScript sources, use CoffeeScript 1.3.3 (newer versions may or may not work properly)

\end{itemize}

\section{Usage}

To open up the simulator, run
\begin{lstlisting}[language=bash]
python serve.py
\end{lstlisting}
to start a simple local webserver and open \url{http://127.0.0.1:8000/simulator.html} in your
browser.
You should get a window with a Semni outline in the top left and some controls and boxes under
neath it. Using your mouse, you can interact with the simulation (for now, you need to be scrolled
all the way up for that to properly work).

\subsection{Data plotters}

The data plotters are in the top right of the page.
To see current data and a history of the last 1000 time steps (96 Hz control loop), click the enable boxes in the top left of each graph. Phase trajectory displays in red, with the hip angle on the x-axis and the knee angle on the y-axis (it might be hard to find if there is no movement). The motor torque is displayed in green for the hip and blue for the knee torque.

\subsection{Buttons, Toggles and Shortcuts}

There are a number of controls in the interface that are grouped by usage.\ The ``Simulation
Controls'' allow the selection of morphologies, at the moment single and double pendulum and Semni.
``Simulate in Realtime'' is intended to make the physics move the robot at about 60 fps which
corresponds to about the same speed as realtime. Otherwise the simulation will run as fast as
possible. ``Pause/Run Simulation'' toggles between halting and running again. The graph layouting and
other processes independent from the physics will not be paused. While paused, progressing one step
can be achieved by pressing ``Next step(s)''. This means however 10 physics simulation steps and one
controller step, c.f. Section~\ref{sim_internal}. The buttons ``toggle CSL'' and ``toggle bounce
controller'' each enable or disable motor controllers for both joints. The CSL controller will use
the paramters below the buttons, the bounce controller (using constant velocity and changing
direction on stall) has default parameters.

The ``ABC learning'' group contains controls related to the learning framework. ``toggle explore'' will
also toggle the CSL controller and will start exploring. It will use the set ``next mode strategy''
setting to determine what next csl mode to set when a fixpoint was detected. This change will be
reflected in the select boxes and the CSL parameters on the left.  The buttons ``Save graph as SVG'',
``Save graph as JSON'' and ``Load graph from JSON'' should be self explanatory. ``Load graph from
Semni'' loads a list of nodes as they are printed over a serial console from the associated ABC
implementation on a hardware Semni robot.  All progress will be displayed in the graph below all of
the controls after postures are detected. The group ``Graph properties'' holds settings for the visual
graph representation. Repulsion and stiffness relate to the spring simulation between nodes and
changes the way the layouting looks and behaves. These will need a browser with support for html
range elements. ``Animate graph'' will toggle whether the graph is continuously redrawn or only once
when a new posture is detected or when the user moves or hovers nodes in the graph with the mouse.
``Pause graph layouting between new poses'' will make the layout algorithm be stopped after a few
seconds of processing whenever a posture was found. Otherwise it will run continuously and might
slow down the physics simulation. If the page was loaded from a remote url, web workers will be used
for the layouting which runs the layouting on a different CPU core than the physics (if possible)
and will make this setting unnecessary. ``Show node activation'' will show the numerical value that
indicates how much there is to explore or to still be learned at each node. This is used with the
unseen mode strategy to determine which edge to follow when for one node, all possible directions
have been taken already. The same information is shown by colour if ``Show node activation colors'' is
activated. ``Show Semni postures'' will trigger the display of small semni images above each node in
the graph. ``Show transition labels'' will show the CSL mode that was set during a transition from one
node to another, in addition to the mode on the node, as these may differ. Finally, ``Save graph with
every new posture'' will save the whole graph as JSON file whenever a new pose is found to have an
automatic history of what happened.  Further details of the learning process can be found in the web
browser's console.

The more frequent buttons have one letter shortcuts as depicted by the [ ] brackets in their button
label, so pressing this letter key will call that action.

Next to the simulation visualization, there are options to zoom the view, show current angles and
center of mass (COM) and drop in test objects. It is also possible to save and load the current
position and angles of the semni robot.

\section{Morphologies}
\subsection{Included}
\begin{itemize}
\item Single Pendulum
\item Double Pendulum
\item Semni
\end{itemize}

\subsection{Custom}

Import Data, Preparation, Creating Box2D entity

\section{Simulator Interna}
\label{sim_internal}

The simulation uses Box2D (2.1a3) ported to JavaScipt and patched by me so a torque
can be applied directly to a joint.
The morphology is drawn on an html5 canvas, the graph is drawn directly with svg.
The html5 range element provides sliders for certain values (make sure your browser supports them).

Apart from the basic physics simulation, there is a simple motor model and a simple fluid friction
model for the joint servos.

\begin{figure}[H]
    \centering
    
\tikzstyle{connector} = [->,thick]

\begin{tikzpicture}[remember picture,
  inner/.style={inner sep=3pt},
  outer/.style={inner sep=10pt},
  state/.style={
         rectangle,
         rounded corners,
         draw=black, very thick,
         minimum height=2em,
         inner sep=0.3cm,
         text centered,
         anchor=center
         }
  ]
  \node[outer,state,label=Controller] (A) {
    \begin{tikzpicture}[node distance=1cm]
        \node[inner,multiplier]                  (m0)  {};
        \textaboveof{m0}{$g_\text{i}$}

        \node[inner,input,below=of m0]          (Ain0) {};
        \textrightof{Ain0}{[$\phi-\phi^{-1}$]}

        \node[inner,adder,right=of m0]           (a0)  {};
        \node[inner,multiplier,above right=0.7cm and 0.3cm of a0] (m1)  {};
        \textrightof{m1}{$g_\text{f}$}

        \node[inner,delay,above left=0.5cm and 0.1cm of a0]       (d0) {$z^{-1}$};
        \node[inner,adder,right=of a0]           (a1) {};
        \node[inner,output,right=of a1]          (Aout) {};
        \textbelowof{Aout}{[$u$]}

        \node[inner,input,below=of a1]           (Ain1) {};
        \textrightof{Ain1}{$g_\text{b}$}

        \draw [connector] (Ain0) to (m0);
        \draw [connector] (m0) to (a0);
        \draw [connector, bend right=45] (a0) to (m1);
        \draw [connector, bend right=45] (m1) to (d0);
        \draw [connector, bend right=45] (d0) to (a0);
        \draw [connector] (a0) to (a1);
        \draw [connector] (a1) to (Aout);
        \draw [connector] (Ain1) to (a1);
    \end{tikzpicture}
  };
  \node[outer,state,label=Motor,right=of A,yshift=-1cm] (B) {
    \begin{tikzpicture}
        \node [inner,input] (Bin0) {};
        \textbelowof{Bin0}{[$u$]};

        \node [inner,adder,right=of Bin0] (sum) {};
        \node [inner,multiplier,right=of sum] (resistor) {};
        \textaboveof{resistor}{$\frac{1}{R}$};

        \node [inner,multiplier,right=of resistor] (constant0) {};
        \textaboveof{constant0}{$k_\text{m}$};

        \node [inner,output,below=of constant0] (Bout) {};
        \textrightof{Bout}{[$M$]};

        \node [inner,multiplier,below=of resistor] (constant1) {};
        \textrightof{constant1}{$k_\text{b}$};

        \node [inner,input,below=of constant1] (Bin1) {};
        \textrightof{Bin1}{[$\dot{\phi}$]};

        \node[rounded corners,draw=black,thick,minimum width=5mm,minimum height=5mm, inner sep=0,left=0.85cm of constant1] (clip) {
            \begin{tikzpicture}
                \draw[semithick] (-1.75mm,-1.25mm) -- (0,-1.25mm);
                \draw[semithick] (0,-1.25mm) -- (1.5mm,1.25mm);
                \draw[semithick] (1.5mm,1.25mm) -- (3.25mm,1.25mm);
            \end{tikzpicture}
        };
        \textbelowof{clip}{\small{[-12V,+12V]}};

        \draw[connector] (Bin0) to node[pos=0.8,above,yshift=-0.15cm]{$\scriptstyle+$} (sum);
        \draw[connector] (sum) to (resistor);
        \draw[connector] (resistor) to (constant0);
        \draw[connector] (constant0) to (Bout);
        \draw[connector] (Bin1) to (constant1);
        \draw[connector] (constant1) to (clip);
        \draw[connector] (clip) to node[pos=0.8,left,xshift=0.2cm]{$\scriptstyle-$} (sum);
    \end{tikzpicture}
  };
  \node[outer,state,below=of B, xshift=-2cm,yshift=0.3cm] (D) { Box2D };
  \node[outer,state,label=Stiction,below=of A,yshift=-2.5cm] (C) {
      \begin{tikzpicture}
          \node [inner,multiplier,anchor=north west] at (0cm, 1cm) (constant) {};
          \textleftof{constant}{$\beta$};

          \node [inner,input,right=of constant] (Cin0) {};
          \textaboveof{Cin0}{[$\dot{\phi}$]};

          \node [inner,output,above=of constant] (Cout) {};
          \textaboveof{Cout}{[$M$]};

          \draw [connector] (Cin0) to (constant);
          \draw [connector] (constant) to (Cout);

      \end{tikzpicture}
  };

  \draw[connector] (Aout) -- (Bin0);
  \draw[connector] (Bout) |- (D.-10);
  \draw[connector] (Cout) -| (D.-110);

  \draw[connector] (D.180) -| (Ain0);
  \draw[connector] (D.10) -| (Bin1);
  \draw[connector] (D.-70) |- (Cin0);

\end{tikzpicture}



    \caption{The components of the simulation CSL controller, motor model and
    friction model and Box2D. Empty circles denote a multiplication, circles with plus sign
    an addition. Plus or minus on an arrow give the sign for an addition.
    Values in square brackets give the symbol for an dimension. In this way, $u$ denotes
    a voltage, $M$ a Torque and $\dot{\phi}$ an angular velocity.}
    \label{fig:controllermodell}
\end{figure}

\subsection{Friction}

In addition to the movements of the objects themselves, frition forces occur between two bodies that
are in contact. These are directed against the forces that produce the motion and can be considered
threefold.  If the two bodies remain completely still, \textit{static friction} occurs, which is
dependent on the materials and the pressure and prevents movement up to a maximum force $F_H^{Max}$.
So it can be though ofa force $F_H$ equal to the moving force $F$ being applied in opposite
direction.  Let $mu_H$ be the friction coefficient of static friction and $F_N$ be the force
perpendicular to the surface. Then for $v = 0$:
\begin{eqnarray}
    F_H & = & -F \\
    F_H \le \vec F_H^{Max} & = & \mu_H \cdot F_N
\end{eqnarray}

If the applied force is greater than $F_H^{Max}$, movement begins along the surfaces and the bodies
are no longer subject to static friction but to \textit{gliding friction}.
This in turn is divided into the \textit {dry friction}, a constant frictional force,
and \textit{viscous friction}, which depends on the velocity of the movement.
The dry friction force is also called \textit{Coulomb friction}.
and is only dependent on the normal force $F_N$ and a coefficient $\mu_G$ for the
specific material combination that is involved.
\begin{eqnarray}
    F_{G_T} = \mu_G \cdot F_N
\end{eqnarray}

The viscous friction force, however, is additionally dependent on velocity $v$ or the
angular velocity $\omega$ respectively and a separate coefficient $\mu_V$.

\begin{eqnarray}
    F_{G_V} = \mu_V \cdot F_N \cdot v
\end{eqnarray}

\end{document}

\section{ABC learning}
    \subsection{CSL modes}
    \subsection{Posture Graph}
